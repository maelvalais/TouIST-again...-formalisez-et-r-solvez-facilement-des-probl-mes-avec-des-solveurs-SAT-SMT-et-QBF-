% !TeX root = IAF2015-TouIST.tex
% la ligne précédente sert pour le logiciel TexWorks afin de pouvoir compiler ce fichier directement sans devoir ouvrir satoulouse_main.tex!
The piece of software \satoulouse is an open-source project globally written in
JAVA so that it can be used on many different platforms.

\subsection{The graphical user interface}


The graphical user interface (GUI) enables the user to write formulas and to
click on the button ``check if the set of formulas is satisfiable''. The GUI
is written in JAVA in order to be able to use the suitable library SWING. In
particular, as this library is object oriented, it was really simple to create
our own widget to enter a formula of propositional logic. Formulas are also
displayed in \LaTeX\ style thanks to the library jLatexMath\footnote{\url{http://forge.scilab.org/index.php/p/jlatexmath/}} which does not require any
additional software (in particular it does not require a real \LaTeX\
installation).


\subsection{The SAT solver}

The SAT solver used as backend of \satoulouse\ is SAT4J
\footnote{\url{http://www.sat4j.org/}}. The first advantage of this solver is
that it is written in JAVA so it is easy to call it from the GUI. The second
one is its good efficiency \cite{berre10:_sat4j}, and even if it is not the best SAT
solver in the world (see SAT Race 2008), it is one of the best. 


\subsection{Connection between the SAT solver and the graphical user interface}

The translation from formulas written by the user to formulas for the SAT
solver SAT4J is divided in two parts:
\begin{itemize}
\item expanding formulas, that is to say, rewrite macros into expanded
  conjunction normal forms. For instance, we expand the formula $\bigwedge_{i
    \in \{1 .. 4\}} p_i$ into $(((p_1 \land p_2) \land p_3) \land
  p_4)$. Expanding formulas is done in Scheme code intepreted with the JAVA
  library kawa \cite{bothner1998kawa}. Scheme is a suitable language to
  perform this translation because parsing of formulas is directly done in
  this language: formulas are directly s-expressions.

\item translating the expanded conjunctive normal forms into a rigorous SAT4J
  input. This part is done in Java because we need to run SAT4J which is coded
  in JAVA. We construct a map between the propositions of the GUI (strings)
  and strict positive integers used by SAT4J and then build corresponding
  arrays of integer for SAT4J.
\end{itemize}



\subsection{Loading and saving a problem}

As we can give explicit names to propositions (and not only integers as for
SAT4J), as we have macros (like $\bigwedge_{i \in \{1 .. 4\}} switch_i$),
\satoulouse\ has its own file format: we simply write s-expressions representing
formulas. Nevertheless, \satoulouse\ is also able to load standard DIMACS cnf
format files as depicted below.\\[0pt]


%\begin{figure}
%\begin{center}
\begin{minipage}{6cm}
\begin{center}
\begin{verbatim}
c A sample .cnf file. 
p cnf 3 2
1 -3 0
2 3 -1 0 
1 2 -3 0
\end{verbatim}
\end{center}
\end{minipage}
\begin{minipage}{6cm}
\mbox{  }\\[18pt]
$((p_1 \vee \neg p_3) \wedge$

$(p_2 \vee p_3 \vee \neg p_1) \wedge$

$ (p_1 \vee p_2 \vee \neg p_3))$s
\end{minipage}
%\end{center}
%\caption{
\begin{center}The standard DIMACS cnf format\label{figure:dimacsformat}\end{center}
%}
%\end{figure}

%\begin{figure}
%%\begin{center}
%\begin{tabular}{lll}
%%\begin{center}
%%\begin{verbatim}
% c A sample .cnf file. & &\\
%p cnf 3 2 & \\
%1 -3 0  & \phantom{xxxxxxxxxxxxxxx} & $((p_1 \vee \neg p_3) \wedge$ \\
%2 3 -1 0  & & $(p_2 \vee p_3 \vee \neg p_1) \wedge$\\
%1 2 -3 0 & & $ (p_1 \vee p_2 \vee \neg p_3))$
%%\end{verbatim}
%%\end{center}
%\end{tabular}
%\caption{The standard DIMACS cnf format\label{figure:dimacsformat}}
%\end{figure}

%%% Local Variables: 
%%% mode: latex
%%% TeX-master: "satoulouse_main"
%%% End: 
