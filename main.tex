\documentclass{iaf}

\annee{2017}

\usepackage[pdftex]{graphicx}
\usepackage{url}
\usepackage{amssymb,amsmath}
\usepackage{xspace}
\usepackage{tikz}
\usepackage[colorlinks,hyperindex,bookmarks,linkcolor=blue,citecolor=blue,urlcolor=blue]{hyperref}
\usepackage{mathrsfs} 
\usepackage{ifthen}
%try
% Typo
\newcommand{\vs}{\emph{vs.}\@\xspace}
\newcommand{\touist}{\textsc{TouIST}\xspace}
\newcommand{\satoulouse}{\textsc{SAToulouse}\xspace}
\newcommand{\guill}[1]{\emph{#1}}
\newcommand{\etc}{\emph{etc.}\@\xspace}
\newcommand{\warning}[1]{\textcolor{red}{#1}}

% sémantique
\newcommand{\interpret}[1][NIL]{%
    \ifthenelse{\equal{#1}{NIL}}{\mathscr{I}}{\mathscr{I}\Big ( #1 \Big )}%
}

% Logique
\newcommand{\limp}{\rightarrow}
\newcommand{\lequiv}{\leftrightarrow}

% macro du \game
\newcommand{\game}{jeu de Nim\xspace}
\newcommand{\nbAllumettes}{\mathit{NA}}
\newcommand{\nbJoueurs}{\mathit{NJ}}
\newcommand{\matchesSet}{A}
\newcommand{\turnsSet}{T}
\newcommand{\turn}[2][0]{\mathit{tour\_de\_}#1(#2)}
\newcommand{\rest}[2]{\mathit{reste}(#1, #2)}
\newcommand{\takes}[2][2]{\mathit{prend\_2}(#1)}
\newcommand{\lost}[1][0]{#1\_\mathit{perd}}

% \newcommand{\ufb}{\mbox{$\stackrel{?}{=}$}} % unifiable
% \newcommand{\FALSE}{\perp}
% \newcommand{\TRUE}{\top}



\titre{\touist again\ldots\\
\small (formalisez et résolvez facilement des problèmes avec des solveurs SAT, SMT et QBF)}

\auteurs{Olivier Gasquet$^\dag$, Andreas Herzig$^\ddag$, Dominique Longin$^\ddag$, Frédéric Maris$^\dag$, Maël Valais$^\dag$}

\institutions{\small IRIT (Institut de Recherche en Informatique de Toulouse)\\
\small $^\ddag$CNRS, $^\dag$Université Paul Sabatier, Toulouse, France}

\mels{\small\{Olivier.Gasquet,Andreas.Herzig,Dominique.Longin,Frederic.Maris,Mael.Valais\}@irit.fr}

\begin{document}



\creationEntete

%\begin{resume} bla bla bla ...

% Les solveurs SAT sont des outils puissants pour résoudre des problèmes logiques de taille réelle, mais leur utilisation nécessite des connaissances solides. Elle peut être vue par rapport à la logique comme l'utilisation d'un langage d'assemblage par rapport à la programmation. Il manque un langage de haut niveau pour permettre à des utilisateurs divers de tirer facilement profit de ces outils. \touist vise à combler cette lacune.

% Il est dédié à la logique propositionnelle et ses principales fonctions sont (1) d'offrir un langage logique de haut niveau pour exprimer succinctement des formules complexes (par exemple des formules décrivant les règles du Sudoku, des problèmes de planification...) et (2) de trouver des modèles à ces formules en utilisant un solveur adéquat et performant, que l'utilisateur n'a pas besoin de connaître.
% Il consiste en une interface conviviale qui propose plusieurs facilités syntaxiques et qui fait appel à des solveurs suffisamment puissants pour permettre de résoudre automatiquement de grandes instances de problèmes difficiles (emplois du temps, Sudokus...). Il peut interagir avec différents démonstrateurs: solveurs SAT pur mais également solveurs SMT (SAT modulo théories - comme la théorie linéaire sur les réels, etc). Il peut donc être utilisé aussi bien par des débutants pour  des problèmes purement propositionnels, que par des étudiants de cycles supérieurs ou même des chercheurs et ingénieurs, par exemple pour résoudre des problèmes de planification impliquant de grands ensembles d'actions et des contraintes numériques.
%\end{resume}

%%%%%%%%%%%%%%%%%%%%%%%%%%%%%%%%%%%%%%%%%%%%%%%%%%%%
%%%%%%%%%%%%%%%%%%%%%%%%%%%%%%%%%%%%%%%%%%%%%%%%%%%%
\section{Introduction}
%%%%%%%%%%%%%%%%%%%%%%%%%%%%%%%%%%%%%%%%%%%%%%%%%%%%
%%%%%%%%%%%%%%%%%%%%%%%%%%%%%%%%%%%%%%%%%%%%%%%%%%%%

Depuis 2010, nous développons \touist\footnote{Historiquement, \touist est le successeur de \satoulouse, présenté pour la première fois lors de la conférence ICTTL'2011 \cite{GaScSt2011}.}, un logiciel dédié à la logique propositionnelle dont les principales fonctionnalités sont (1) d'offrir un langage logique de haut niveau pour exprimer succinctement des formules complexes et (2) de trouver des modèles à ces formules en utilisant un solveur SAT performant. 

Dans ce qui suit, nous présentons une extension de \touist à QBF (\emph{Quantified Boolean Formulas}) au travers d'un exemple : le \game.

Tout d'abord, nous survolons succinctement les principales caractéristiques de \touist (Section~\ref{sec:touistDescription}) et montrons comment modéliser le \game dans le langage d'entrée de \touist (Section~\ref{sec:gameDescription}). Enfin, après une brève présentation de QBF nous montrons comment modéliser la recherche d'une stratégie gagnante dans \touist\ pour ce jeu (Section~\ref{sec:QBFandTouist}).



%%%%%%%%%%%%%%%%%%%%%%%%%%%%%%%%%%%%%%%%%%%%%%%%%%%%
%%%%%%%%%%%%%%%%%%%%%%%%%%%%%%%%%%%%%%%%%%%%%%%%%%%%
\section{Présentation générale de \touist}
%%%%%%%%%%%%%%%%%%%%%%%%%%%%%%%%%%%%%%%%%%%%%%%%%%%%
%%%%%%%%%%%%%%%%%%%%%%%%%%%%%%%%%%%%%%%%%%%%%%%%%%%%
\label{sec:touistDescription}
\input{Files/1-touistDescription.tex}

%%%%%%%%%%%%%%%%%%%%%%%%%%%%%%%%%%%%%%%%%%%%%%%%%%%%
%%%%%%%%%%%%%%%%%%%%%%%%%%%%%%%%%%%%%%%%%%%%%%%%%%%%
\section{Description du \game}
%%%%%%%%%%%%%%%%%%%%%%%%%%%%%%%%%%%%%%%%%%%%%%%%%%%%
%%%%%%%%%%%%%%%%%%%%%%%%%%%%%%%%%%%%%%%%%%%%%%%%%%%%
\label{sec:gameDescription}
\input{Files/2-gameDescription.tex}

%%%%%%%%%%%%%%%%%%%%%%%%%%%%%%%%%%%%%%%%%%%%%%%%%%%%
%%%%%%%%%%%%%%%%%%%%%%%%%%%%%%%%%%%%%%%%%%%%%%%%%%%%
\section{Formalisation d'une stratégie gagnante à l'aide de QBF}
%%%%%%%%%%%%%%%%%%%%%%%%%%%%%%%%%%%%%%%%%%%%%%%%%%%%
%%%%%%%%%%%%%%%%%%%%%%%%%%%%%%%%%%%%%%%%%%%%%%%%%%%%
\label{sec:QBFandTouist}
\input{Files/3-QBFandTouist.tex}

%%%%%%%%%%%%%%%%%%%%%%%%%%%%%%%%%%%%%%%%%%%%%%%%%%%%
%%%%%%%%%%%%%%%%%%%%%%%%%%%%%%%%%%%%%%%%%%%%%%%%%%%%
\section{Conclusion}
%%%%%%%%%%%%%%%%%%%%%%%%%%%%%%%%%%%%%%%%%%%%%%%%%%%%
%%%%%%%%%%%%%%%%%%%%%%%%%%%%%%%%%%%%%%%%%%%%%%%%%%%%

\touist peut être vu comme un compilateur de langages logiques étendus et de haut niveau vers des prouveurs efficaces indépendants. Ces deux facettes lui confèrent une grande facilité d'utilisation, un large spectre d'application et de bonnes performances calculatoires. À ce titre, il constitue un outil complètement original et unique en son genre. 

Nous l'utilisons dans le cadre du cours d'initiation à la logique en % va t'en
licence d'informatique, mais aussi en master, dans le cadre de travaux pratiques et de projets. Les étudiants sont ainsi appelés à parcourir tout le processus allant de la formalisation à la résolution de problèmes qui vont au-delà des problèmes-jouets faisables sur papier. 

Mais plus encore, \touist\ est d'ores et déjà  utilisé par des chercheurs dans le cadre de travaux mené au sein de notre laboratoire et impliquant une modélisation logique (planification, raisonnement épistémique via traduction en QBF,\ldots), il comble un manque existant au sein des logiciels de calculs formels comme Maple, SageMath, Mathematica ou Maxima qui n'intègrent qu'anecdotiquement des outils logiques.


\appendix
\input{Files/4-Appendix.tex}

% Bibliographie en utilisant BibTeX
\bibliography{biblio}


\end{document}


